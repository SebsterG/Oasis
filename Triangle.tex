\documentclass[a4paper,norsk]{article}
\usepackage[latin1]{inputenc}
\usepackage[T1]{fontenc}
\usepackage{babel,textcomp,listings, subfigure,graphicx}
                                    
\title{Triangle Cavity Flow}
\author{Sebastian Gjertsen}
\begin{document}
\maketitle
\begin{center}
\section*{Introduction}
In this study I have looked at steady 2-D incompressible flow inside a cavity driven triangle. This seemingly simple geometry gives interesting flows for different Reynolds number, and the implementation is rather easy to check with published papers. I used [Erturk and Gokcol] and [Spectral element book]  as a reference to make mesh and to look at formation of eddies. The goal will be to compare results with [Erturk and Gokcol] and [Spectral element book] , and discuss results.
\end{center}
\section*{Numerics}
I used the FEniCS software with Oasis, which solves the Navier-Stokes equations both transient and steady. I set the u(x,y) = (1,0) at the top and set no slip on the sides.
To look at eddies forming I computed the stream function:
$$ \nabla^2 \psi = -\omega, \hspace{4mm} \omega = \nabla \times  u $$
where I set $\psi = 0 $ on the boundary. \newline
The streamfunction was programmed the following way:
\begin{lstlisting}
    bc0 = DirichletBC(V, (0, 0), nos)
    bc1 = DirichletBC(V, (1.0, 0), top)
    uu = project(u_, V, bcs=[bc0,bc1])
    psi = Function(Q)
    p = TrialFunction(Q)
    q = TestFunction(Q)
    solve(inner(grad(p), grad(q))*dx == inner(curl(u_), q)*dx, psi,\
     bcs=[DirichletBC(Q, 0, "on_boundary")])
    sort = psi.vector().array().argsort()
    print pa,"--",sort[0], sort[1]
    xx = interpolate(Expression("x[0]"), Q)
    yy = interpolate(Expression("x[1]"), Q)
    xm_1 = xx.vector()[sort[0]]
    ym_1 = yy.vector()[sort[0]]
    xm_2 = xx.vector()[sort[-1]]
    ym_2 = yy.vector()[sort[-1]]
    print "Center main-eddy: x: %.4f, %.4f " %(xm_1, ym_1)
    print "Stream function value at main-eddy: %.4f " %(psi(xm_1,ym_1))
    mycurl = project(curl(u_), Q, bcs=[DirichletBC(Q, 0, "on_boundary")])
    v_value = mycurl(xm_1,ym_1)
    print "Vorticity value at main-eddy: %.4f "%(v_value)\end{lstlisting}



\section*{Results)}
First I considered an equilateral Triangle with corner points: \newline
a = (-$\sqrt{3}$, 1), b = ($\sqrt{3}$, 1),  c = (0, -2) \newline
\begin{figure}
    \centering
    \includegraphics[trim = 25mm 0mm 25mm 0mm, clip, scale=0.4]{Equilateral_Re_100.png}
    \caption{Streamfunction of equilateral triangle}
    \label{fig:awesome_image}
\end{figure}
\newline
Where the Reynolds number is defined as $Re = \frac{U}{\mu} $, where I set $U=1$ and controlled Re by changing $\mu$
\newline
Table~\ref{table:equilateral} looks at the position of the center eddy and the value of the streamfunction and vorticity at this point. \newline
This study was done with 73541 vertices and 145664 cells. \newline
We see that most of the streamfunction values, vorticity and position are similar to $10^{-2}$. As we increase Re the streamfunction value stays very similar, but the vorticity starts to become a little different but similar to $10^{-1}$. 
The position of the main eddy is very similar to Re = 1250, and changes a bit at Re = 1500.
\newline
\begin{table}
  \centering%
  \begin{tabular}{l*{6}{c}r}
   & &Erturk and Gockol & & & Me\\
   \hline 
  Re & $\psi$ & $\omega$ &(x,y) & $\psi$ & $\omega$ & (x,y)  \\
  1   &   -0.2329  &-1.3788  & (0.0101, 04668) &-0.2329 & -1.3658  & (0.0109, 0.4617)  \\
  50   &  -0.2369 & -1.4689 & (0.3484, 04434)  & -0.2367 &-1.4708 & (0.3525, 0.4467)    \\
  100  & -0.2482 & -1.3669 & (0.3315, 0.3555)   & -0.2476 & -1.3599 &(0.3339, 0.3576)  \\
  200   & -0.2624 &-1.2518  & (0.2030, 0.2734)  & -0.2613 & -1.2459 & (0.1978, 0.2758 ) \\
  350   & -0.2724 & -1.1985 & (0.1556, 0.2383) &-0.2708 & -1.1887 & (0.1605, 0.2412)   \\
  500   & -0.2774 & -1.1791 &  (0.1319, 0.2207)  & -0.2754 &  -1.1666 & (0.1395, 0.2237)       \\
  750   &  -0.2818 & -1.1668 &  (0.1150, 0.2031) & -0.2793  & -1.1504  & (0.1160, 0.2056 ) \\
  1000   & -0.2844 & -1.1629 &(0.1116, 0.1973) & -0.2814  & -1.1427 & (0.1066, 0.1944) \\
  1250   & -0.2861 & -1.1624 & (0.1049, 0.1973) & -0.2826 & -1.1382 & (0.1066, 0.1944)   \\
  1500   & -0.2872 & -1.1639 & (0.1015, 0.1914)  & -0.2834 & -1.1354  & (0.1022, 0.1828)\\
  1750   &  -0.2881 & -1.1675 & (0.1015, 0.1914) &   &\\ 
\hline
\end{tabular}
  \caption{Eddies of equilateral Triangle}\label{table:equilateral}
\end{table}


Next I considered an Isoceles Triangle with corner points: \newline
a = (-1, 0), b = (1, 0) ,c = (0, -4), with 64051 vertices and 126444 cells\newline
Again I looked at the position of the main eddy in Table~\ref{table:isoceles_triangle}.
\newline
\begin{figure}
    \centering
    \includegraphics[trim = 100mm 0mm 100mm 0mm, clip, scale=0.4]{isoceles_Re_100.png}
    \caption{Streamfunction of isoceles triangle, Re = 100}
     \includegraphics[trim = 100mm 0mm 100mm 0mm, clip, scale=0.4]{isoceles_Re_200.png}
     \caption{Streamfunction of isoceles triangle, Re = 200}
    \label{fig:awesome_image}
\end{figure}


\begin{table}
  \centering%
  \begin{tabular}{l*{6}{c}r}
   & Erturk and Gockol & Me\\
   \hline 
   Re &(x,y) & (x,y)  \\
   12.5 & (0.059, -0.391) & (0.0587, -0.4001 )\\ 
   25 & (0.115,-0.398) &  (0.1101, -0.3999 )\\
   100 & (0.213, -0.477)& (0.2124, -0.4767 ) \\
   200 & (0.129, -0.563)& (0.1275, -0.5608 )\\  
   \hline
  \end{tabular}
  \caption{Position of main eddy, isoceles triangle}\label{table:isoceles_triangle}
\end{table}

The next thing i looked at was where the second eddy was. This eddy is spinning the opposite direction so i had to look for the highest value of the stream function:
\newline
\begin{table}
  \centering%
  \begin{tabular}{l*{6}{c}r}
   & Me\\
   \hline 
  	Re & $\psi$ & $\omega$& (x,y)   \\
	12.5 & 0.0002 & 0.0108 & (-0.0018, -2.1924)\\
	100 & 0.0020 & 0.0699  & (0.0490, -1.8199)\\
	200 &  0.0073 & 0.2183 & (-0.0237, -1.6809)\\
	\hline
 \end{tabular}
  \caption{Position of second eddy,isoceles }\label{table:second eddy}
\end{table}
\newline
As we see in figure 2 and 3 the second eddy moved up and to the left, which correlates with table 3.
\newline

The next thing i looked at was how many eddies i could find. According to [Spectral] every eddy formed will be 406 times weaker then the previous. So the velocities and the very bottom eddies will be very small. To find these eddies I looked at the absolute velocities in the horizontal direction on a line straight through the mesh from the top at (0.0) to (0,-4). As we can see when the plot has a dip is where the velocities have changed direction and where we can find the eddies center. I used a very fine mesh at the bottom to catch as many eddies as a could. After a careful count, i got 9 eddies. This was done with Re=1, and with a mesh consisting of 57843 vertices and 113176 cells.\newline
We can see that near the bottom we get disturbances when the eddies becomes to small for the mesh. One could argue that with a fine enough mesh the number of eddies would go to infinity.
\newline
If we compare to [Spectral] in figure 5, where they have used a reynolds number of 0 (Stokes flow), we see that we get very similar results.
\begin{figure}
    \centering
    \includegraphics[trim = 0mm 0mm 0mm 0mm, clip, scale=0.4]{eddy_plot_1.png}
    \caption{Absolute transverse velocity along the centre line}
    \includegraphics[trim = 0mm 0mm 0mm 0mm, clip, scale=0.4]{eddy_plot_2.png}
    \caption{Absolute transverse velocity along lower part of the centre line}
    \includegraphics[trim = 50mm 10mm 10mm 70mm, clip, scale=0.4]{Spectral_eddy_plot.png}
    \caption{Absolute transverse velocity, from [Spectral book]}
    \label{fig:awesome_image}
\end{figure}
\newline





\end{document}